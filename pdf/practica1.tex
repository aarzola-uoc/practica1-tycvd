\documentclass[]{article}
\usepackage{lmodern}
\usepackage{amssymb,amsmath}
\usepackage{ifxetex,ifluatex}
\usepackage{fixltx2e} % provides \textsubscript
\ifnum 0\ifxetex 1\fi\ifluatex 1\fi=0 % if pdftex
  \usepackage[T1]{fontenc}
  \usepackage[utf8]{inputenc}
\else % if luatex or xelatex
  \ifxetex
    \usepackage{mathspec}
  \else
    \usepackage{fontspec}
  \fi
  \defaultfontfeatures{Ligatures=TeX,Scale=MatchLowercase}
\fi
% use upquote if available, for straight quotes in verbatim environments
\IfFileExists{upquote.sty}{\usepackage{upquote}}{}
% use microtype if available
\IfFileExists{microtype.sty}{%
\usepackage{microtype}
\UseMicrotypeSet[protrusion]{basicmath} % disable protrusion for tt fonts
}{}
\usepackage[margin=1in]{geometry}
\usepackage{hyperref}
\hypersetup{unicode=true,
            pdftitle={Práctica 1: Web scraping},
            pdfauthor={Miguel Santos Pérez y Alejandro Arzola García},
            pdfborder={0 0 0},
            breaklinks=true}
\urlstyle{same}  % don't use monospace font for urls
\usepackage{longtable,booktabs}
\usepackage{graphicx}
% grffile has become a legacy package: https://ctan.org/pkg/grffile
\IfFileExists{grffile.sty}{%
\usepackage{grffile}
}{}
\makeatletter
\def\maxwidth{\ifdim\Gin@nat@width>\linewidth\linewidth\else\Gin@nat@width\fi}
\def\maxheight{\ifdim\Gin@nat@height>\textheight\textheight\else\Gin@nat@height\fi}
\makeatother
% Scale images if necessary, so that they will not overflow the page
% margins by default, and it is still possible to overwrite the defaults
% using explicit options in \includegraphics[width, height, ...]{}
\setkeys{Gin}{width=\maxwidth,height=\maxheight,keepaspectratio}
\IfFileExists{parskip.sty}{%
\usepackage{parskip}
}{% else
\setlength{\parindent}{0pt}
\setlength{\parskip}{6pt plus 2pt minus 1pt}
}
\setlength{\emergencystretch}{3em}  % prevent overfull lines
\providecommand{\tightlist}{%
  \setlength{\itemsep}{0pt}\setlength{\parskip}{0pt}}
\setcounter{secnumdepth}{5}
% Redefines (sub)paragraphs to behave more like sections
\ifx\paragraph\undefined\else
\let\oldparagraph\paragraph
\renewcommand{\paragraph}[1]{\oldparagraph{#1}\mbox{}}
\fi
\ifx\subparagraph\undefined\else
\let\oldsubparagraph\subparagraph
\renewcommand{\subparagraph}[1]{\oldsubparagraph{#1}\mbox{}}
\fi

%%% Use protect on footnotes to avoid problems with footnotes in titles
\let\rmarkdownfootnote\footnote%
\def\footnote{\protect\rmarkdownfootnote}

%%% Change title format to be more compact
\usepackage{titling}

% Create subtitle command for use in maketitle
\providecommand{\subtitle}[1]{
  \posttitle{
    \begin{center}\large#1\end{center}
    }
}

\setlength{\droptitle}{-2em}

  \title{Práctica 1: Web scraping}
    \pretitle{\vspace{\droptitle}\centering\huge}
  \posttitle{\par}
  \subtitle{UOC - Tipología y ciclo de vida de los datos}
  \author{Miguel Santos Pérez y Alejandro Arzola García}
    \preauthor{\centering\large\emph}
  \postauthor{\par}
      \predate{\centering\large\emph}
  \postdate{\par}
    \date{14 de abril de 2020}


\begin{document}
\maketitle

{
\setcounter{tocdepth}{2}
\tableofcontents
}
\newpage

\hypertarget{contexto}{%
\section{Contexto}\label{contexto}}

El auge del valor del dato y su aplicación a cualquier ámbito de la
sociedad es claro desde su explosión unos años atrás. En este sentido,
cada vez más tecnologías basadas en el poder del dato se emplean con
eficiencia en distintos deportes como el fútbol, para diversas tareas
tales como el fichaje de jugadores, campañas de marketing o estrategias
de partidos. Técnicas similares son aplicadas al tenis.

En cuanto al deporte del que nos ocuparemos en este trabajo,
concretamente el baloncesto vivimos en un mundo bipolar. Mientras en
América, las poderosas franquicias NBA cada vez lo utilizan más en su
día a día con grandes equipos de Data Scientist, en Europa el gasto en
la explotación del dato se mantiene en segundo plano. Así pues, la
inspiración de este estudio es desarrollar el webscrapping sobre el
conjunto de datos de baloncesto en Europa, como primera aproximación
para luego utilizar estos datos y buscarles su valor. En este primer
trabajo, por tanto, se tratará de acceder a los partidos de Euroliga y
almacernarlos en datasets que sean analizables.

Todos estos resultados se encuentran en la web oficial de la
competición, www.euroleague.net

\includegraphics{C:/Users/USER-PC/Desktop/logo.PNG =100x20}

\hypertarget{componentes-del-grupo}{%
\section{Componentes del grupo}\label{componentes-del-grupo}}

\begin{itemize}
\tightlist
\item
  Miguel Santos Pérez
  (\href{mailto:miguel8santos@uoc.edu}{\nolinkurl{miguel8santos@uoc.edu}})
\item
  Alejandro Arzola García
  (\href{mailto:aarzola@uoc.edu}{\nolinkurl{aarzola@uoc.edu}})
\end{itemize}

\hypertarget{repositorio-github}{%
\section{Repositorio Github}\label{repositorio-github}}

Para la realización de esta práctica se ha creado un repositorio en
\emph{GitHub} para trabajar de manera colaborativa y tener un control de
versiones sobre el código fuente. Se puede acceder a este repositorio a
través del siguiente enlace:

\begin{itemize}
\tightlist
\item
  \url{https://github.com/aarzola-uoc/practica1-tycvd}
\end{itemize}

\hypertarget{tuxedtulo-para-el-dataset}{%
\section{Título para el dataset}\label{tuxedtulo-para-el-dataset}}

Euroliga 2019-2020. Marcadores y estadísticas por partido.

\hypertarget{descripciuxf3n-del-dataset}{%
\section{Descripción del dataset}\label{descripciuxf3n-del-dataset}}

Se propone para este trabajo la obtención de dos datasets, el primero
con todos los marcadores, partidos y enlaces a estadísticas y, el
segundo, con el detalle de las propias estadísticas para su posterior
análisis.

\hypertarget{representaciuxf3n-gruxe1fica}{%
\section{Representación gráfica}\label{representaciuxf3n-gruxe1fica}}

El proyecto representa, para cada partido, en un dataset los siguientes
datos.\\
El primer dataset, recoge los resultados y es el siguiente:

\begin{figure}
\centering
\includegraphics{C:/Users/USER-PC/Desktop/resultado.png}
\caption{Dataset 1}
\end{figure}

El segundo dataset, recoge las estadísticas individuales dentro de un
partido y representa:

\includegraphics{C:/Users/USER-PC/Desktop/estadistica1.png}
\includegraphics{C:/Users/USER-PC/Desktop/estadistica2.png}

\hypertarget{contenido}{%
\section{Contenido}\label{contenido}}

Así, definimos la estructura de los dataset como:

\begin{itemize}
\tightlist
\item
  \textbf{Euroleague\_Scoreboards}: Dicho dataset recoge las resultados
  por partido y se compone de siete campos:

  \begin{itemize}
  \tightlist
  \item
    MatchId: identificador único del partido.
  \item
    Date: fecha del partido.
  \item
    HomeTeam: nombre del equipo local.
  \item
    HomeScore: puntos anotados por el equipo local
  \item
    VisitingTeam: nombre del equipo visitante.
  \item
    VisitingScore: puntos anotados por el equipo visitante.
  \item
    Link: link de estadísticas.
  \end{itemize}
\item
  \textbf{Euroleague\_Stats\_per\_Game}: Dicho dataset recoge las
  estadísticas a nivel jugador por partido y se compone de:

  \begin{itemize}
  \tightlist
  \item
    MatchId: identificador único de partido.
  \item
    Team: equipo al que pertenece el jugador.
  \item
    PlayerNumber: número del jugador.
  \item
    PlayerName: nombre del jugador.
  \item
    Min: minutos jugados.
  \item
    Pts: puntos anotados.
  \item
    2FG: tiros de dos (metidos-anotados).
  \item
    3FG: tiros de tres (metidos-anotados).
  \item
    FT: tiros libres (metidos-anotados).
  \item
    O: rebotes ofensivos.
  \item
    D: rebotes defensivos.
  \item
    T: rebotes totales.
  \item
    As: asistencias.
  \item
    St: recuperaciones.
  \item
    To: pérdidas.
  \item
    Fv: tapones a favor.
  \item
    Ag: tapones en contra.
  \item
    Cm: faltas cometidas.
  \item
    Rv: faltas recibidas.
  \item
    PIR: valoración del jugador.
  \end{itemize}
\end{itemize}

El periodo de recolección de los datos es esta temporada, desde los
partidos qeu empezaron el 3 de octubre hasta los terminados el 5 de
marzo.

\hypertarget{publicaciuxf3n-del-dataset}{%
\section{Publicación del dataset}\label{publicaciuxf3n-del-dataset}}

Se han publicados los dos \emph{datasets} obtenidos en el repositorio
web \textbf{Zenodo} y se pueden acceder a través del siguiente enlace:

\begin{itemize}
\tightlist
\item
  \url{https://zenodo.org/record/3740661\#.XoouwVNKgnU}
\end{itemize}

El DOI (\emph{Digital Object Identifier}) asignado ha sido el siguiente:

\begin{itemize}
\tightlist
\item
  10.5281/zenodo.3740661
\end{itemize}

\hypertarget{agradecimientos}{%
\section{Agradecimientos}\label{agradecimientos}}

Los datos, recogidos de la web oficial de la competición, son propiedad
de © Euroleague Ventures SA. Para ello se ha hecho uso del lenguaje de
programación Python y de técnicas de WebScrapping para extraer la
información alojada en las páginas.

\hypertarget{inspiraciuxf3n}{%
\section{Inspiración}\label{inspiraciuxf3n}}

De la mano de lo explicado en la contextualización, se hace interesante
la disponibilidad de los datos de una manera sencilla de analizar como
primer paso de cara a un posterior uso en el planteamiento de los
partidos o de las semanas de entrenamiento. La mayor ambición es la
realización de un análisis profundo que permita a los equipos disponer
de una herramienta que les proporcione en los datos un valor añadido. A
partir de las estadísticas recogidas por Euroliga, se podrían crear
nuevas estadísticas más precisas que recojan otro tipo de datos en
función de los ya existentes.

\hypertarget{licencia}{%
\section{Licencia}\label{licencia}}

\hypertarget{contribuciones-al-trabajo}{%
\section{Contribuciones al trabajo}\label{contribuciones-al-trabajo}}

\begin{longtable}[]{@{}lc@{}}
\toprule
\textbf{Contribuciones} & \textbf{Firma}\tabularnewline
\midrule
\endhead
Investigación previa & MSP, AAG\tabularnewline
Redacción de las respuestas & MSP, AAG\tabularnewline
Creación del repositorio GitHub & MSP, AAG\tabularnewline
Desarrollo código & MSP, AAG\tabularnewline
Publicación de los datasets & MSP, AAG\tabularnewline
\bottomrule
\end{longtable}


\end{document}
